\chapter{How data is transferred (atomically)}
\label{chapter.slide}
\svnrev{r4750}

A data transfer from a node to one of its (two) neighbours is also called a
\emph{slide}. A slide operation is defined in the \erlmodule{slide\_op} module,
the protocol is mainly implemented in \erlmodule{dht\_node\_move}.
Parts of the slide are dependent on the ring maintenance implementation and are
split off into modules implementing the \erlmodule{slide\_beh} behaviour.

Though the protocols are mainly symmetric, we distinguish between sending data
to the predecessor and sending data to the successor, respectively. In the
following protocol visualisations, arrows denote message exchanges, pseudo-code
for operations that are being executed is put at the side of each time bar.
Functions in green are those implemented in the \erlmodule{slide\_beh}
behaviour, if annotated with an arrow pointing to itself, this callback is
asynchronous.
During the protocol, the slide operation goes through several phases which are
show in black boxes.

In general, a slide consists of three steps:
\begin{enumerate}
 \item set up slide
 \item send data \& start recording changes, i.e. delta
 \item send delta \& transfer responsibility
\end{enumerate}

The latter two may be repeated to execute incremental slides which further
reduce periods of unavailability. During this period, no node is responsible
for the range to transfer and messages are thus delayed until the receiving
node gains responsibility.

%\pagebreak
\section{Sending data to the predecessor}

\subsection{Protocol}
\documentclass[a4paper]{scrreprt}
\usepackage{typearea}
\areaset[1cm]{165mm}{240mm}

\usepackage[T1]{fontenc}
\usepackage[latin1]{inputenc}
\usepackage{listings}

\usepackage{xcolor}
\definecolor{lightyellow}{rgb}{1.0, 1.0, 0.5}
\definecolor{codebackground}{HTML}{EEEEEE}
\definecolor{commandinput}{rgb}{0.8,0.8,1}
\definecolor{lightblue}{HTML}{1E90FF}

\usepackage{tikz}
\usetikzlibrary{positioning}
\usetikzlibrary{shadows}
\usetikzlibrary{fit}
\usetikzlibrary{shapes.arrows}
\usetikzlibrary{backgrounds}
\usepackage[graphics,tightpage,active]{preview}
\PreviewEnvironment{tikzpicture}
\newlength{\imagewidth}
\newlength{\imagescale}

\usepackage{calc}

\newcommand{\code}[1]{\lstinline[basicstyle=\ttfamily]!#1!}

\begin{document}


\newlength{\tikzinnerheight}
\setlength{\tikzinnerheight}{1.0cm}
\newlength{\tikzsepheight}
\setlength{\tikzsepheight}{0.75cm}
\newlength{\tikzstartsecond}
\setlength{\tikzstartsecond}{0cm}
\newlength{\tikzstartfirst}
\setlength{\tikzstartfirst}{\tikzstartsecond + \tikzinnerheight + 0.5\tikzsepheight}
\newlength{\tikzsepinner}
\setlength{\tikzsepinner}{\tikzinnerheight + \tikzsepheight}
% \newlength{\tikzendsucc}
% \setlength{\tikzendsucc}{\tikzsepinner + \tikzstartfirst - \tikzstartsecond}

\begin{tikzpicture}
 [pre/.style={<-,shorten <=1pt,>=stealth,semithick},
  post/.style={->,shorten >=1pt,>=stealth,semithick},
  progress/.style={-,dashed,thin,black},
  timeout/.style={draw=black!50, dashed},
  process/.style={rectangle,black,rounded corners},
  start/.style={rectangle,draw,thick,fill=yellow,draw=black,minimum height=0.7cm},
  inner/.style={minimum height=\tikzinnerheight,thin},
  end/.style={minimum height=0.4cm},
  phase/.style={rectangle,draw,thick,black},
  my_node/.style={rectangle,fill=codebackground,drop shadow,rounded corners},
  action_l/.style={rectangle,black,font=\footnotesize,align=right},
  action_r/.style={rectangle,black,font=\footnotesize,align=left},
  note/.style={circle, thin, draw, outer sep=0.1cm},
  async_r/.style={post,min distance=0.5,looseness=2.5,in=0,out=0},
  async_l/.style={post,min distance=0.5,looseness=2.5,in=180,out=180},
  async_desc/.style={rectangle,black,font=\footnotesize},
  msg/.style={sloped},
  msg_t/.style={msg,anchor=south},
  msg_b/.style={msg,anchor=north},
  bend angle=60]

 \node[font={\Large\bfseries}] (heading1) {Send data to predecessor};
 \node[font=\footnotesize,below=-0.2 of heading1] (heading2) {(version 2.0)};

 \node[start] (pred)      [below left=0.5 and 2.5 of heading1.south] {pred};
 \node[inner] (pred-init)           [below=\tikzstartfirst of pred]      {};
 \node[inner] (pred-got-data)       [below=\tikzsepinner of pred-init] {};
 \node[inner] (pred-got-delta)      [below=\tikzsepinner of pred-got-data] {};
 \node[inner] (pred-got-owner)      [below=\tikzsepinner of pred-got-delta] {};
 \node[end]   (pred-end)  [below=\tikzstartsecond of pred-got-owner] {\footnotesize pred};

 \node[start] (succ)      [below right=0.5 and 2.5 of heading1.south] {succ};
 \node[inner] (succ-init)           [below=\tikzstartsecond of succ]      {};
 \node[inner] (succ-send-data)      [below=\tikzsepinner of succ-init]      {};
 \node[inner] (succ-send-delta)     [below=\tikzsepinner of succ-send-data]      {};
 \node[inner] (succ-send-owner)     [below=\tikzsepinner of succ-send-delta]      {};
 \node[end]   (succ-end)  [below=\tikzstartfirst of succ-send-owner] {\footnotesize succ};

 \path[-] (pred)
            edge [progress] (pred-end)
          (succ)
            edge [progress] (succ-end);

 \path[->] (succ-init.south west)
             edge [post,draw=lightgray] node[msg_t, lightgray,font=\footnotesize] {slide, pred, 'send'} node[msg_b, lightgray,font=\footnotesize] {(optional)} (pred-init.north east)
           (pred-init.south east)
             edge [post]  node[msg_t] {slide, succ, 'rcv'\textcolor{lightgray}{, MaxE}} (succ-send-data.north west)

           (succ-send-data.east)
             edge [async_r] node[async_desc, auto, anchor=west] {\textcolor{green}{prepare\_send\_data(SlideOp)}}
                  (succ-send-data.south east)

           (succ-send-data.south west)
             edge [post] node[msg_t] {data\textcolor{lightgray}{, TargetId, NextOp}} (pred-got-data.north east)
           (pred-got-data.south east)
             edge [post] node[msg_t] {data\_ack} (succ-send-delta.north west)

           (pred-got-data.north west)
             edge [async_l] node[async_desc, auto, anchor=east] {\textcolor{green}{update\_rcv\_data(SlideOp, TargetId, NextOp)}}
                  (pred-got-data.west)

           (succ-send-delta.north east)
             edge [async_r] node[async_desc, auto, anchor=west] {\textcolor{green}{prepare\_send\_selta(SlideOp)}}
                  (succ-send-delta.east)

           (succ-send-delta.south west)
             edge [post] node[msg_t] {delta} (pred-got-delta.north east)
           (pred-got-delta.south east)
             edge [post] node[msg_t] {delta\_ack} (succ-send-owner.north west)

           (succ-send-owner.north east)
             edge [async_r] node[async_desc, auto, anchor=west] (update_owner) {\textcolor{green}{update\_owner(SlideOp)}}
                  (succ-send-owner.east);


 \node[action_l, left=0.1 of pred-init.east] {
   SlideOp.new()\\
   fd.subscribe(SlideOp.node)\\
   \textcolor{green}{prepare\_rcv\_data(SlideOp)}%
 };
 \node[phase, below left=0.35 and 0.1 of pred-init.south] (pred-init-p-leases) {wait\_for\_data};

 \node[phase, below left=0.1 and 0.1 of pred-got-data.south] {wait\_for\_delta};

 \node[action_l, left=0.1 of pred-got-delta.east] {
   \textcolor{green}{finish\_delta(SlideOp)}\\
   fd.unsubscribe(SlideOp.node)\\
   SlideOp.delete()%
 };

 \node[action_l, left=0.1 of pred-got-owner.east] {}; % nothing to do



 \node[action_r, right=0.1 of succ-init.south west, lightgray] {
   SlideOp.new()\\
   fd.subscribe(SlideOp.node)%
 };
 \node[phase, below right=0.4 and 0.1 of succ-init.south, lightgray] {wait\_for\_other};

 \node[action_r, right=0.1 of succ-send-data.north west,anchor=west] {
   SlideOp.new()\\
   fd.subscribe(SlideOp.node)%
 };

 \node[action_r, below=0.3 of succ-send-data.south,anchor=west] {
   db.record\_changes(SlideOp.interval)%
 };
 \node[phase, below right=0.5 and 0.1 of succ-send-data.south] {wait\_for\_data\_ack};

 \node[action_r, below right=0.1 and 0.1 of succ-send-delta.south west, anchor=south west] {
   db.stop\_record\_changes(SlideOp.interval)%
 };
 \node[phase, below right=0.1 and 0.1 of succ-send-delta.south] {wait\_for\_delta\_ack};

 \coordinate (succ-cont-start1) at ($(succ-send-owner)-(0,.5\tikzsepinner)$);
 \coordinate (succ-cont-end) at ($(succ-send-data)+(0,0.1)$);
 \coordinate (pred-cont-start1) at ($(pred-got-delta.south)+(0,-0.25)$);
 \coordinate (pred-cont-start2) at ($(pred-cont-start1)-(0,1.25)$);
 \coordinate (pred-cont-end) at ($(pred-init.south)+(0,-0.25)$);

 \path[->] (update_owner.south west)
             edge [async_r,draw=lightgray,dashed,out=-50,in=150,looseness=0.8] node[msg_t, pos=0.5, lightgray, font=\footnotesize] {rm\_update} node[msg_b, pos=0.5, lightgray, font=\footnotesize] {(to dht\_node)} node[pos=0.37,note,draw,red,anchor=north] {$\ell$} ($(pred-cont-start1)!0.7!(pred-cont-start2)$);

 \draw[post, rounded corners, dashed, lightgray] (succ-cont-start1) -- +(7,0) -- ($(succ-cont-end)+(7,0)$) -- (succ-cont-end);
 \node[action_l, lightgray, anchor=south east] at ($(succ-cont-start1)+(7,0)$) {if (NextOp == continue)\\SlideOp.update()};

 \node[action_r, below right=\tikzstartfirst and 0.1 of succ-send-owner.north west, anchor=north west] {
   fd.unsubscribe(SlideOp.node)\\
   SlideOp.delete()%
 };

 \coordinate (pred-cont-mid) at ($(pred-cont-start1)+(-7.5,0.2)$);
 \draw[post, rounded corners, dashed, lightgray] (pred-cont-start1) -- +(-7.5,0) -- ($(pred-cont-end)+(-7.5,0)$) -- (pred-cont-end);
 \draw[post, rounded corners, dashed, lightgray] (pred-cont-start2) -- +(-7.5,0) -- (pred-cont-mid);
 \node[action_r, lightgray, anchor=west] at ($(pred-cont-start1)!0.5!(pred-cont-start2)+(-7.5,0)$) {if (NextOp == continue) -> continue\\SlideOp.update()\\prepare\_rcv\_data(SlideOp)};

\end{tikzpicture}
\end{document}


\subsection{Callbacks}
% \medskip
{%\small
\begin{tabular}{P{4.2cm}P{5.4cm}P{5.4cm}}
  \toprule
  & \code{slide_chord}
  & \code{slide_leases} \tn
  \midrule
  %
  \bfseries $\leftarrow$ prepare\_rcv\_data
  & \emph{\color{gray}nothing to do}
  & \emph{\color{gray}nothing to do} \tn
  \midrule
  %
  \bfseries $\rightarrow$ prepare\_send\_data
  & add DB range
  & \emph{\color{gray}nothing to do} \tn
  \midrule
  %
  \bfseries $\leftarrow$ update\_rcv\_data
  & set MSG forward,\\change my ID
  & \emph{\color{gray}nothing to do} \tn
  \midrule
  %
  \bfseries $\rightarrow$ prepare\_send\_delta
  & wait until pred up-to-date,\\then: remove DB range
  & split own lease into two ranges, locally disable lease sent to pred \tn
  \midrule
  %
  \bfseries $\leftarrow$ finish\_delta
  & remove MSG forward
  & \emph{\color{gray}nothing to do} \tn
  \midrule
  %
  \bfseries $\rightarrow$ finish\_delta\_ack
  & \emph{\color{gray}nothing to do}
  & hand over the lease to pred, notify pred of owner change \tn
  \bottomrule
\end{tabular}
}
% \medskip

\pagebreak
\section{Sending data to the successor}

\subsection{Protocol}
\documentclass[a4paper]{scrreprt}
\usepackage{typearea}
\areaset[1cm]{165mm}{240mm}

\usepackage[T1]{fontenc}
\usepackage[latin1]{inputenc}
\usepackage{listings}

\usepackage{xcolor}
\definecolor{lightyellow}{rgb}{1.0, 1.0, 0.5}
\definecolor{codebackground}{HTML}{EEEEEE}
\definecolor{commandinput}{rgb}{0.8,0.8,1}
\definecolor{lightblue}{HTML}{1E90FF}

\usepackage{tikz}
\usetikzlibrary{positioning}
\usetikzlibrary{shadows}
\usetikzlibrary{fit}
\usetikzlibrary{shapes.arrows}
\usetikzlibrary{backgrounds}
\usepackage[graphics,tightpage,active]{preview}
\PreviewEnvironment{tikzpicture}
\newlength{\imagewidth}
\newlength{\imagescale}

\usepackage{calc}

\newcommand{\code}[1]{\lstinline[basicstyle=\ttfamily]!#1!}

\begin{document}


\newlength{\tikzinnerheight}
\setlength{\tikzinnerheight}{1.0cm}
\newlength{\tikzsepheight}
\setlength{\tikzsepheight}{0.75cm}
\newlength{\tikzstartsecond}
\setlength{\tikzstartsecond}{0cm}
\newlength{\tikzstartfirst}
\setlength{\tikzstartfirst}{\tikzstartsecond + \tikzinnerheight + 0.5\tikzsepheight}
\newlength{\tikzsepinner}
\setlength{\tikzsepinner}{\tikzinnerheight + \tikzsepheight}
% \newlength{\tikzendsucc}
% \setlength{\tikzendsucc}{\tikzsepinner + \tikzstartfirst - \tikzstartsecond}

\begin{tikzpicture}
 [pre/.style={<-,shorten <=1pt,>=stealth,semithick},
  post/.style={->,shorten >=1pt,>=stealth,semithick},
  progress/.style={-,dashed,thin,black},
  timeout/.style={draw=black!50, dashed},
  process/.style={rectangle,black,rounded corners},
  start/.style={rectangle,draw,thick,fill=yellow,draw=black,minimum height=0.7cm},
  inner/.style={minimum height=\tikzinnerheight,thin},
  end/.style={minimum height=0.4cm},
  phase/.style={rectangle,draw,thick,black},
  my_node/.style={rectangle,fill=codebackground,drop shadow,rounded corners},
  action_l/.style={rectangle,black,font=\footnotesize,align=right},
  action_r/.style={rectangle,black,font=\footnotesize,align=left},
  note/.style={circle, thin, draw, outer sep=0.1cm},
  async_r/.style={post,min distance=0.5,looseness=2.5,in=0,out=0},
  async_l/.style={post,min distance=0.5,looseness=2.5,in=180,out=180},
  async_desc/.style={rectangle,black,font=\footnotesize},
  msg/.style={sloped},
  msg_t/.style={msg,anchor=south},
  msg_b/.style={msg,anchor=north},
  bend angle=60]

 \node[font={\Large\bfseries}] (heading1) {Send data to successor};
 \node[font=\footnotesize,below=-0.2 of heading1] (heading2) {(version 2.0)};

 \node[start] (pred)      [below left=0.5 and 2.5 of heading1.south] {pred};
 \node[inner] (pred-init)           [below=\tikzstartsecond of pred]      {};
 \node[inner] (pred-send-data)      [below=\tikzsepinner of pred-init] {};
 \node[inner] (pred-send-delta)     [below=\tikzsepinner of pred-send-data] {};
 \node[inner] (pred-send-owner)     [below=\tikzsepinner of pred-send-delta] {};
 \node[end]   (pred-end)  [below=\tikzstartfirst of pred-send-owner] {\footnotesize pred};

 \node[start] (succ)      [below right=0.5 and 2.5 of heading1.south] {succ};
 \node[inner] (succ-init)           [below=\tikzstartfirst of succ]      {};
 \node[inner] (succ-got-data)      [below=\tikzsepinner of succ-init]      {};
 \node[inner] (succ-got-delta)     [below=\tikzsepinner of succ-got-data]      {};
 \node[inner] (succ-got-owner)      [below=\tikzsepinner of succ-got-delta] {};
 \node[end]   (succ-end)  [below=\tikzstartsecond of succ-got-owner] {\footnotesize succ};

 \path[-] (pred)
            edge [progress] (pred-end)
          (succ)
            edge [progress] (succ-end);

 \path[->] (pred-init.south east)
             edge [post,lightgray]  node[msg_t,lightgray,font=\footnotesize] {slide, succ, 'send'} node[msg_b,lightgray,font=\footnotesize] {(optional)} (succ-init.north west)
           (succ-init.south west)
             edge [post] node[msg_t] {slide, pred, 'rcv'\textcolor{lightgray}{, MaxE}} (pred-send-data.north east)

           (pred-send-data.west)
             edge [async_l] node[async_desc, auto, anchor=east, align=right] {\textcolor{green}{prepare\_send\_data(SlideOp)}}
                  (pred-send-data.south west)

           (pred-send-data.south east)
             edge [post] node[msg_t] {data\textcolor{lightgray}{, TargetId, NextOp}} (succ-got-data.north west)
           (succ-got-data.south west)
             edge [post] node[msg_t] {data\_ack} (pred-send-delta.north east)

           (pred-send-delta.north west)
             edge [async_l] node[async_desc, auto, anchor=east] {\textcolor{green}{prepare\_send\_selta(SlideOp)}}
                  (pred-send-delta.west)

           (pred-send-delta.south east)
             edge [post] node[msg_t] {delta} (succ-got-delta.north west)
           (succ-got-delta.south west)
             edge [post] node[msg_t] {delta\_ack} (pred-send-owner.north east)

           (pred-send-owner.north west)
             edge [async_l] node[async_desc, auto, anchor=east] (update_owner) {\textcolor{green}{update\_owner(SlideOp)}}
                  (pred-send-owner.west);


 \node[action_l, left=0.1 of pred-init.south east ,lightgray] {
   SlideOp.new()\\
   fd.subscribe(SlideOp.node)%
 };
 \node[phase, below left=0.4 and 0.1 of pred-init.south, lightgray] {wait\_for\_other};

 \node[action_l, left=0.1 of pred-send-data.north east] {
   SlideOp.new()\\
   fd.subscribe(SlideOp.node)%
 };
 
 \node[action_l, left=0.1 of pred-send-data.south east, anchor=north east] {
   db.record\_changes(SlideOp.interval)%
 };
 \node[phase, below left=0.5 and 0.1 of pred-send-data.south] {wait\_for\_data\_ack};
 
 \node[action_l, below left=0.1 and 0.1 of pred-send-delta.south east, anchor=south east] {
   db.stop\_record\_changes(SlideOp.interval)%
 };
 \node[phase, below left=0.1 and 0.1 of pred-send-delta.south] {wait\_for\_delta\_ack};


 
 \node[action_r, right=0.1 of succ-init.west] {
   SlideOp.new()\\
   fd.subscribe(SlideOp.node)\\
   \textcolor{green}{prepare\_rcv\_data(SlideOp)}%
 };
 \node[phase, below right=0.35 and 0.1 of succ-init.south] {wait\_for\_data};

 \node[action_r, right=0.1 of succ-got-data.west] {
   \textcolor{green}{update\_rcv\_data(SlideOp, TargetId, NextOp)}%
 };
 \node[phase, below right=0.1 and 0.1 of succ-got-data.south] {wait\_for\_delta};

 \node[action_r, right=0.1 of succ-got-delta.west] {
   \textcolor{green}{finish\_delta(SlideOp)}%
 };

 \node[action_r, right=0.1 of succ-got-owner.west] {}; % nothing to do

 \coordinate (succ-cont-start1) at ($(succ-got-delta.south)-(0,0.25)$);
 \coordinate (succ-cont-start2) at ($(succ-cont-start1)-(0,1.25)$);
 \coordinate (succ-cont-end) at ($(succ-init.south)+(0,-0.25)$);
 \coordinate (pred-cont-start1) at ($(pred-send-owner)-(0,.5\tikzsepinner)$);
 \coordinate (pred-cont-end) at ($(pred-send-data.west)+(0,0.1)$);

 \path[->] (update_owner.south east)
             edge [async_l,draw=lightgray,dashed,out=230,in=30,looseness=0.8] node[msg_t, pos=0.5, lightgray, font=\footnotesize] {rm\_update} node[msg_b, pos=0.5, lightgray, font=\footnotesize] {(to dht\_node)} node[pos=0.37,note,draw,red,anchor=north] {$\ell$} ($(succ-cont-start1)!0.7!(succ-cont-start2)$);

 \draw[post, rounded corners, dashed, lightgray] (pred-cont-start1) -- +(-7,0) -- ($(pred-cont-end)+(-7,0)$) -- (pred-cont-end);
 \node[action_r, lightgray, anchor=south west] at ($(pred-cont-start1)+(-7,0)$) {if (NextOp == continue)\\SlideOp.update()};

 \node[action_l, below left=\tikzstartfirst and 0.1 of pred-send-owner.north east, anchor=north east] {
   fd.unsubscribe(SlideOp.node)\\
   SlideOp.delete()%
 };
 
 \coordinate (succ-cont-mid) at ($(succ-cont-start1)+(7,0.2)$);
 \draw[post, rounded corners, dashed, lightgray] (succ-cont-start1) -- +(7,0) -- ($(succ-cont-end)+(7,0)$) -- (succ-cont-end);
 \draw[post, rounded corners, dashed, lightgray] (succ-cont-start2) -- +(7,0) -- (succ-cont-mid);
 \node[action_l, lightgray, anchor=east] at ($(succ-cont-start1)!0.5!(succ-cont-start2)+(7,0)$) {if (NextOp == continue) -> continue\\SlideOp.update()\\prepare\_rcv\_data(SlideOp)};

 \node[action_r, below right=\tikzstartsecond and 0.1 of succ-got-owner.north west, anchor=north west] {
   fd.unsubscribe(SlideOp.node)\\
   SlideOp.delete()%
 };

\end{tikzpicture}
\end{document}


\subsection{Callbacks}

% \medskip
{%\small
\begin{tabular}{P{4.2cm}P{5.4cm}P{5.4cm}}
  \toprule
  & \code{slide_chord}
  & \code{slide_leases} \tn
  \midrule
  %
  \bfseries $\rightarrow$ prepare\_rcv\_data
  & set MSG forward
  & \emph{\color{gray}nothing to do} \tn
  \midrule
  %
  \bfseries $\leftarrow$ prepare\_send\_data
  & add DB range,\\change my ID
  & \emph{\color{gray}nothing to do} \tn
  \midrule
  %
  \bfseries $\rightarrow$ update\_rcv\_data
  & \emph{\color{gray}nothing to do}
  & \emph{\color{gray}nothing to do} \tn
  \midrule
  %
  \bfseries $\leftarrow$ prepare\_send\_delta
  & remove DB range
  & split own lease into two ranges, locally disable lease sent to succ \tn
  \midrule
  %
  \bfseries $\rightarrow$ finish\_delta
  & remove MSG forward,\\until pred up-to-date:\\$\hookrightarrow$ add DB range\\then: remove DB range
  & \emph{\color{gray}nothing to do} \tn
  \midrule
  %
  \bfseries $\leftarrow$ finish\_delta\_ack
  & \emph{\color{gray}nothing to do}
  & hand over the lease to succ, notify succ of owner change \tn
  \bottomrule
\end{tabular}
}
% \medskip
