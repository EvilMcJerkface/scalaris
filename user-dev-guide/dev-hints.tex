\chapter{General Hints}

\section{Coding Guidelines}

\begin{itemize}
\item Keep the code short
\item Use \erlmodule{gen\_component} to implement additional processes
\item Don't use receive by yourself (Exception: to implement single threaded
  user API calls (cs\_api, yaws\_calls, etc)
\item Don't use \erlfun{erlang}{now}{/0}, \erlfun{erlang}{send\_after}{/3},
  \code{receive after} etc. in performance critical code, consider
  using \erlmodule{msg\_delay} instead.
\item Don't use \erlfun{timer}{tc}{/3} as it catches exceptions. Use
  \erlfun{util}{tc}{/3} instead.
\end{itemize}

\section{Testing Your Modifications and Extensions}

\begin{itemize}
\item Run the testsuites using \code{make test}
\item Run the java api test using \code{make java-test}
      (\scalaris{} output will be printed if a test fails; if you want to see
      it during the tests, start a \code{bin/firstnode.sh} and run the tests by
      \code{cd java; ant test})
\item Run the Ruby client by starting \scalaris{} and running
      \code{cd contrib; ./jsonrpc.rb}
\end{itemize}

\section{Help with Digging into the System}
\label{sec:digging}

\begin{itemize}
\item use \erlfun{ets}{i}{/0,1} to get details on the local state of some
      processes
\item consider changing pdb.erl to use ets instead of erlang:put/get
\item Have a look at strace -f -p PID of beam process
\item Get message statistics via the Web-interface
\item enable/disable tracing for certain modules
%% \item enable gen-component profiling and read the collected data using
%% 
%%    \code{lists:reverse(lists:keysort(2,ets:tab2list(profiling))).}
%% 
%%    (Has the limitation that it measures using absolute time, maybe the data
%%    is more ok, when using -smp disable? ...)
\item Use etop and look at the total memory size and atoms generated
\item send processes sleep or kill messages to test certain behaviour (see
  \erlmodule{gen\_component}.erl)
\item use \code{mgmt_server:number_of_nodes(). flush().}
\item use \code{admin_checkring(). flush().}
\end{itemize}

%% \section{General Erlang server loop}
%% 
%% Servers in Erlang often use the following structure to maintain a state
%% while processing received messages:
%% 
%% \lstset{language=erlang}
%% \begin{lstlisting}
%% loop(State) ->
%%   receive
%%     Message ->
%%       State1 = f(State),
%%       loop(State1)
%%   end.
%% \end{lstlisting}
%% 
%% The server runs an endless loop, that waits for a message, processes it and
%% calls itself using tail-recursion in each branch. The loop works on a
%% \code{State}, which can be modified when a message is handled.
