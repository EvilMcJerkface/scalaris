\chapter{Testing the system}

\section{Erlang unit tests}
There are some unit tests in the \code{test} directory which test \scalaris{}
itself (the Erlang code). You can call them
by running \code{make test} in the main directory. The results are stored
in a local \code{index.html} file. 

The tests are implemented with the \code{common-test} package from the
Erlang system. For running the tests we rely on \code{run\_test},
which is part of the \code{common-test} package, but (on erlang $<$ R14) is not
installed by default. \code{configure} will check whether \code{run\_test} is
available. If it is not installed, it will show a warning and a short
description of how to install the missing file.

Note: for the unit tests, we are setting up and shutting down several
overlay networks. During the shut down phase, the runtime environment
will print extensive error messages. These error messages do not
indicate that tests failed! Running the complete test suite takes
about 10-20 minutes, depending on your machine.

If the test suite is interrupted before finishing, the results may not have
been linked into the \code{index.html} file. They are however stored in the
\code{ct_run.ct@...} directory.

\section{Java unit tests}
The Java unit tests can be run by executing \code{make java-test} in the main
directory. This will start a \scalaris{} node with the default ports and test
all functions of the Java API. A typical run will look like the
following:

\begin{lstlisting}[language={}]
%> make java-test
[...]
tools.test:
    [junit] Running de.zib.tools.PropertyLoaderTest
    [junit] Testsuite: de.zib.tools.PropertyLoaderTest
    [junit] Tests run: 3, Failures: 0, Errors: 0, Skipped: 0, Time elapsed: 0.017 sec
    [junit] Tests run: 3, Failures: 0, Errors: 0, Skipped: 0, Time elapsed: 0.017 sec
    [junit] 
    [junit] ------------- Standard Output ---------------
    [junit] Working Directory = <scalarisdir>/java-api/classes
    [junit] ------------- ---------------- ---------------
[...]
scalaris.test:
    [junit] Running de.zib.scalaris.ConnectionTest
    [junit] Testsuite: de.zib.scalaris.ConnectionTest
    [junit] Tests run: 7, Failures: 0, Errors: 0, Skipped: 0, Time elapsed: 0.303 sec
    [junit] Tests run: 7, Failures: 0, Errors: 0, Skipped: 0, Time elapsed: 0.303 sec
    [junit] 
    [junit] Running de.zib.scalaris.DefaultConnectionPolicyTest
    [junit] Testsuite: de.zib.scalaris.DefaultConnectionPolicyTest
    [junit] Tests run: 12, Failures: 0, Errors: 0, Skipped: 0, Time elapsed: 0.309 sec
    [junit] Tests run: 12, Failures: 0, Errors: 0, Skipped: 0, Time elapsed: 0.309 sec
    [junit] 
    [junit] Running de.zib.scalaris.ErlangValueTest
    [junit] Testsuite: de.zib.scalaris.ErlangValueTest
    [junit] Tests run: 19, Failures: 0, Errors: 0, Skipped: 0, Time elapsed: 14.444 sec
    [junit] Tests run: 19, Failures: 0, Errors: 0, Skipped: 0, Time elapsed: 14.444 sec
    [junit] 
    [junit] Running de.zib.scalaris.MonitorTest
    [junit] Testsuite: de.zib.scalaris.MonitorTest
    [junit] Tests run: 10, Failures: 0, Errors: 0, Skipped: 0, Time elapsed: 0.064 sec
    [junit] Tests run: 10, Failures: 0, Errors: 0, Skipped: 0, Time elapsed: 0.064 sec
    [junit] 
    [junit] Running de.zib.scalaris.PeerNodeTest
    [junit] Testsuite: de.zib.scalaris.PeerNodeTest
    [junit] Tests run: 5, Failures: 0, Errors: 0, Skipped: 0, Time elapsed: 0.066 sec
    [junit] Tests run: 5, Failures: 0, Errors: 0, Skipped: 0, Time elapsed: 0.066 sec
    [junit] 
    [junit] Running de.zib.scalaris.ReplicatedDHTTest
    [junit] Testsuite: de.zib.scalaris.ReplicatedDHTTest
    [junit] Tests run: 6, Failures: 0, Errors: 0, Skipped: 0, Time elapsed: 0.723 sec
    [junit] Tests run: 6, Failures: 0, Errors: 0, Skipped: 0, Time elapsed: 0.723 sec
    [junit] 
    [junit] Running de.zib.scalaris.ScalarisTest
    [junit] Testsuite: de.zib.scalaris.ScalarisTest
    [junit] Tests run: 7, Failures: 0, Errors: 0, Skipped: 0, Time elapsed: 0.063 sec
    [junit] Tests run: 7, Failures: 0, Errors: 0, Skipped: 0, Time elapsed: 0.063 sec
    [junit] 
    [junit] Running de.zib.scalaris.ScalarisVMTest
    [junit] Testsuite: de.zib.scalaris.ScalarisVMTest
    [junit] Tests run: 42, Failures: 0, Errors: 0, Skipped: 2, Time elapsed: 0.699 sec
    [junit] Tests run: 42, Failures: 0, Errors: 0, Skipped: 2, Time elapsed: 0.699 sec
    [junit] 
    [junit] Testcase: testKillVM1(de.zib.scalaris.ScalarisVMTest):SKIPPED: we still need the Scalaris Erlang VM
    [junit] Testcase: testShutdownVM1(de.zib.scalaris.ScalarisVMTest):SKIPPED: we still need the Scalaris Erlang VM
    [junit] Running de.zib.scalaris.TransactionSingleOpTest
    [junit] Testsuite: de.zib.scalaris.TransactionSingleOpTest
    [junit] Tests run: 34, Failures: 0, Errors: 0, Skipped: 0, Time elapsed: 3.996 sec
    [junit] Tests run: 34, Failures: 0, Errors: 0, Skipped: 0, Time elapsed: 3.996 sec
    [junit] 
    [junit] Running de.zib.scalaris.TransactionTest
    [junit] Testsuite: de.zib.scalaris.TransactionTest
    [junit] Tests run: 30, Failures: 0, Errors: 0, Skipped: 0, Time elapsed: 0.803 sec
    [junit] Tests run: 30, Failures: 0, Errors: 0, Skipped: 0, Time elapsed: 0.803 sec
    [junit] 

test:

BUILD SUCCESSFUL
Total time: 27 seconds
'jtest_boot@csr-pc40.zib.de'
\end{lstlisting}

\section{Python2 unit tests}
The Python unit tests can be run by executing \code{make python-test} in the
main directory. This will start a \scalaris{} node with the default ports and test
all functions of the Python API. A typical run will look like the
following:

\begin{lstlisting}[language={}]
%> make python-test
[...]
testDelete1 (__main__.TestReplicatedDHT) ... ok
testDelete2 (__main__.TestReplicatedDHT) ... ok
testDelete_notExistingKey (__main__.TestReplicatedDHT) ... ok
testDoubleClose (__main__.TestReplicatedDHT) ... ok
testReplicatedDHT1 (__main__.TestReplicatedDHT) ... ok
testReplicatedDHT2 (__main__.TestReplicatedDHT) ... ok
testAddNodes0 (__main__.TestScalarisVM)
Test method for ScalarisVM.addNodes(0). ... ok
testAddNodes1 (__main__.TestScalarisVM)
Test method for ScalarisVM.addNodes(1). ... ok
testAddNodes3 (__main__.TestScalarisVM)
Test method for ScalarisVM.addNodes(3). ... ok
testAddNodes_NotConnected (__main__.TestScalarisVM)
Test method for ScalarisVM.addNodes() with a closed connection. ... ok
testDoubleClose (__main__.TestScalarisVM) ... ok
testGetInfo1 (__main__.TestScalarisVM)
Test method for ScalarisVM.getInfo(). ... ok
testGetInfo_NotConnected (__main__.TestScalarisVM)
Test method for ScalarisVM.getInfo() with a closed connection. ... ok
testGetNodes1 (__main__.TestScalarisVM)
Test method for ScalarisVM.getNodes(). ... ok
testGetNodes_NotConnected (__main__.TestScalarisVM)
Test method for ScalarisVM.getNodes() with a closed connection. ... ok
testGetNumberOfNodes1 (__main__.TestScalarisVM)
Test method for ScalarisVM.getVersion(). ... ok
testGetNumberOfNodes_NotConnected (__main__.TestScalarisVM)
Test method for ScalarisVM.getNumberOfNodes() with a closed connection. ... ok
testGetOtherVMs1 (__main__.TestScalarisVM)
Test method for ScalarisVM.getOtherVMs(1). ... ok
testGetOtherVMs2 (__main__.TestScalarisVM)
Test method for ScalarisVM.getOtherVMs(2). ... ok
testGetOtherVMs3 (__main__.TestScalarisVM)
Test method for ScalarisVM.getOtherVMs(3). ... ok
testGetOtherVMs_NotConnected (__main__.TestScalarisVM)
Test method for ScalarisVM.getOtherVMs() with a closed connection. ... ok
testGetVersion1 (__main__.TestScalarisVM)
Test method for ScalarisVM.getVersion(). ... ok
testGetVersion_NotConnected (__main__.TestScalarisVM)
Test method for ScalarisVM.getVersion() with a closed connection. ... ok
testKillNode1 (__main__.TestScalarisVM)
Test method for ScalarisVM.killNode(). ... ok
testKillNode_NotConnected (__main__.TestScalarisVM)
Test method for ScalarisVM.killNode() with a closed connection. ... ok
testKillNodes0 (__main__.TestScalarisVM)
Test method for ScalarisVM.killNodes(0). ... ok
testKillNodes1 (__main__.TestScalarisVM)
Test method for ScalarisVM.killNodes(1). ... ok
testKillNodes3 (__main__.TestScalarisVM)
Test method for ScalarisVM.killNodes(3). ... ok
testKillNodesByName0 (__main__.TestScalarisVM)
Test method for ScalarisVM.killNodesByName(0). ... ok
testKillNodesByName1 (__main__.TestScalarisVM)
Test method for ScalarisVM.killNodesByName(1). ... ok
testKillNodesByName3 (__main__.TestScalarisVM)
Test method for ScalarisVM.killNodesByName(3). ... ok
testKillNodesByName_NotConnected (__main__.TestScalarisVM)
Test method for ScalarisVM.killNodesByName() with a closed connection. ... ok
testKillNodes_NotConnected (__main__.TestScalarisVM)
Test method for ScalarisVM.killNodes() with a closed connection. ... ok
testScalarisVM1 (__main__.TestScalarisVM) ... ok
testScalarisVM2 (__main__.TestScalarisVM) ... ok
testShutdownNode1 (__main__.TestScalarisVM)
Test method for ScalarisVM.shutdownNode(). ... ok
testShutdownNode_NotConnected (__main__.TestScalarisVM)
Test method for ScalarisVM.shutdownNode() with a closed connection. ... ok
testShutdownNodes0 (__main__.TestScalarisVM)
Test method for ScalarisVM.shutdownNodes(0). ... ok
testShutdownNodes1 (__main__.TestScalarisVM)
Test method for ScalarisVM.shutdownNodes(1). ... ok
testShutdownNodes3 (__main__.TestScalarisVM)
Test method for ScalarisVM.shutdownNodes(3). ... ok
testShutdownNodesByName0 (__main__.TestScalarisVM)
Test method for ScalarisVM.shutdownNodesByName(0). ... ok
testShutdownNodesByName1 (__main__.TestScalarisVM)
Test method for ScalarisVM.shutdownNodesByName(1). ... ok
testShutdownNodesByName3 (__main__.TestScalarisVM)
Test method for ScalarisVM.shutdownNodesByName(3). ... ok
testShutdownNodesByName_NotConnected (__main__.TestScalarisVM)
Test method for ScalarisVM.shutdownNodesByName() with a closed connection. ... ok
testShutdownNodes_NotConnected (__main__.TestScalarisVM)
Test method for ScalarisVM.shutdownNodes() with a closed connection. ... ok
testAbort_Empty (__main__.TestTransaction) ... ok
testAbort_NotConnected (__main__.TestTransaction) ... ok
testCommit_Empty (__main__.TestTransaction) ... ok
testCommit_NotConnected (__main__.TestTransaction) ... ok
testDoubleClose (__main__.TestTransaction) ... ok
testRead_NotConnected (__main__.TestTransaction) ... ok
testRead_NotFound (__main__.TestTransaction) ... ok
testReqList1 (__main__.TestTransaction) ... ok
testReqList_Empty (__main__.TestTransaction) ... ok
testReqTooLarge (__main__.TestTransaction) ... ok
testTransaction1 (__main__.TestTransaction) ... ok
testTransaction3 (__main__.TestTransaction) ... ok
testVarious (__main__.TestTransaction) ... ok
testWriteList1 (__main__.TestTransaction) ... ok
testWriteString (__main__.TestTransaction) ... ok
testWriteString_NotConnected (__main__.TestTransaction) ... ok
testWriteString_NotFound (__main__.TestTransaction) ... ok
testDoubleClose (__main__.TestTransactionSingleOp) ... ok
testRead_NotConnected (__main__.TestTransactionSingleOp) ... ok
testRead_NotFound (__main__.TestTransactionSingleOp) ... ok
testReqList1 (__main__.TestTransactionSingleOp) ... ok
testReqList_Empty (__main__.TestTransactionSingleOp) ... ok
testReqTooLarge (__main__.TestTransactionSingleOp) ... ok
testTestAndSetList1 (__main__.TestTransactionSingleOp) ... ok
testTestAndSetList2 (__main__.TestTransactionSingleOp) ... ok
testTestAndSetList_NotConnected (__main__.TestTransactionSingleOp) ... ok
testTestAndSetList_NotFound (__main__.TestTransactionSingleOp) ... ok
testTestAndSetString1 (__main__.TestTransactionSingleOp) ... ok
testTestAndSetString2 (__main__.TestTransactionSingleOp) ... ok
testTestAndSetString_NotConnected (__main__.TestTransactionSingleOp) ... ok
testTestAndSetString_NotFound (__main__.TestTransactionSingleOp) ... ok
testTransactionSingleOp1 (__main__.TestTransactionSingleOp) ... ok
testTransactionSingleOp2 (__main__.TestTransactionSingleOp) ... ok
testWriteList1 (__main__.TestTransactionSingleOp) ... ok
testWriteList2 (__main__.TestTransactionSingleOp) ... ok
testWriteList_NotConnected (__main__.TestTransactionSingleOp) ... ok
testWriteString1 (__main__.TestTransactionSingleOp) ... ok
testWriteString2 (__main__.TestTransactionSingleOp) ... ok
testWriteString_NotConnected (__main__.TestTransactionSingleOp) ... ok

----------------------------------------------------------------------
Ran 84 tests in 3.565s

OK
'jtest_boot@csr-pc40.zib.de'
\end{lstlisting}

\section{Python3 unit tests}
The Python 3 tests are similar to the Python 2 tests above and can be run by
executing \code{make python3-test}.

\section{Ruby unit tests}
The Ruby unit tests can be run by executing \code{make ruby-test} in the
main directory. This will start a \scalaris{} node with the default ports and test
all functions of the Ruby API. A typical run will look like the
following:

\begin{lstlisting}[language={}]
%> make ruby-test
[...]
# Running tests:

TestReplicatedDHT#testDelete1 = 0.19 s = .
TestReplicatedDHT#testDelete2 = 0.29 s = .
TestReplicatedDHT#testDelete_notExistingKey = 0.05 s = .
TestReplicatedDHT#testDoubleClose = 0.00 s = .
TestReplicatedDHT#testReplicatedDHT1 = 0.00 s = .
TestReplicatedDHT#testReplicatedDHT2 = 0.00 s = .
TestTransaction#testAbort_Empty = 0.00 s = .
TestTransaction#testAbort_NotConnected = 0.00 s = .
TestTransaction#testCommit_Empty = 0.00 s = .
TestTransaction#testCommit_NotConnected = 0.00 s = .
TestTransaction#testDoubleClose = 0.00 s = .
TestTransaction#testRead_NotConnected = 0.00 s = .
TestTransaction#testRead_NotFound = 0.00 s = .
TestTransaction#testReqList1 = 0.02 s = .
TestTransaction#testReqList_Empty = 0.00 s = .
TestTransaction#testReqTooLarge = 0.38 s = .
TestTransaction#testTransaction1 = 0.00 s = .
TestTransaction#testTransaction3 = 0.00 s = .
TestTransaction#testVarious = 0.01 s = .
TestTransaction#testWriteList1 = 0.08 s = .
TestTransaction#testWriteString = 0.11 s = .
TestTransaction#testWriteString_NotConnected = 0.00 s = .
TestTransaction#testWriteString_NotFound = 0.00 s = .
TestTransactionSingleOp#testDoubleClose = 0.00 s = .
TestTransactionSingleOp#testRead_NotConnected = 0.00 s = .
TestTransactionSingleOp#testRead_NotFound = 0.00 s = .
TestTransactionSingleOp#testReqList1 = 0.03 s = .
TestTransactionSingleOp#testReqList_Empty = 0.00 s = .
TestTransactionSingleOp#testReqTooLarge = 0.38 s = .
TestTransactionSingleOp#testTestAndSetList1 = 0.07 s = .
TestTransactionSingleOp#testTestAndSetList2 = 0.05 s = .
TestTransactionSingleOp#testTestAndSetList_NotConnected = 0.00 s = .
TestTransactionSingleOp#testTestAndSetList_NotFound = 0.00 s = .
TestTransactionSingleOp#testTestAndSetString1 = 0.06 s = .
TestTransactionSingleOp#testTestAndSetString2 = 0.08 s = .
TestTransactionSingleOp#testTestAndSetString_NotConnected = 0.00 s = .
TestTransactionSingleOp#testTestAndSetString_NotFound = 0.00 s = .
TestTransactionSingleOp#testTransactionSingleOp1 = 0.00 s = .
TestTransactionSingleOp#testTransactionSingleOp2 = 0.00 s = .
TestTransactionSingleOp#testWriteList1 = 0.06 s = .
TestTransactionSingleOp#testWriteList2 = 0.02 s = .
TestTransactionSingleOp#testWriteList_NotConnected = 0.00 s = .
TestTransactionSingleOp#testWriteString1 = 0.08 s = .
TestTransactionSingleOp#testWriteString2 = 0.05 s = .
TestTransactionSingleOp#testWriteString_NotConnected = 0.00 s = .


Finished tests in 2.040348s, 22.0551 tests/s, 675.8650 assertions/s.

45 tests, 1379 assertions, 0 failures, 0 errors, 0 skips

ruby -v: ruby 2.1.3p242 (2014-09-19 revision 47630) [x86_64-linux-gnu]
'jtest_boot@csr-pc40.zib.de'
\end{lstlisting}

\section{Interoperability Tests}
In order to check whether the common types described in
Section~\sieheref{chapter.systemuse.apis} are fully supported by the APIs
and yield to the appropriate types in another API, we implemented some
interoperability tests. Two make targets exist:
\begin{itemize}
  \item \code{make interop-test} verifies compliance in Java, Python2 and Ruby,
  \item \code{make interop3-test} verifies compliance in Java, Python2, Python3 and Ruby.
\end{itemize}
This will start a \scalaris{} node with the default ports, write test data using
the mentioned APIs and let each API read the data it wrote
itself as well as the data the other APIs wrote. On success it will print

\begin{lstlisting}[language={}]
%> make interop3-test
[...]
all tests successful
\end{lstlisting}
